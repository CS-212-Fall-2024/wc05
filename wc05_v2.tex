\documentclass[a4paper]{exam}

\usepackage{amsmath,amssymb, amsthm}
\usepackage[a4paper]{geometry}
\usepackage{hyperref}
\usepackage{mdframed}
\usepackage{arabtex}
\usepackage{utf8}
\setcode{utf8}


\title{Weekly Challenge 05: Pumping Lemma 2.0}
\author{CS 212 Nature of Computation\\Habib University}
\date{Fall 2024}

\theoremstyle{definition}
\newtheorem{definition}{Definition}

\theoremstyle{claim}
\newtheorem{claim}{Claim}

\qformat{{\large\bf \thequestion. \thequestiontitle}\hfill}
\boxedpoints

% \printanswers %uncomment this

\begin{document}
\maketitle

\begin{questions}
    \titledquestion{Meaningful connections}
    During your time at Habib you have studied graphs in various courses. Along with that you have studied various representations of graphs. One representations that is widely used is the \href{https://en.wikipedia.org/wiki/Adjacency_matrix}{adjacency matrix representation}.
    In your previous CS courses you have encountered flattening a matrix, that is representing a matrix in $\mathbb{R}^{n\times m}$ as a vector in $\mathbb{R}^{n m}$. Now a graph $G = (V, E)$ is said to be connected is for every pair of vertices $v,u \in V$ there is a path between $v$ and $u$ in $G$. Now each finite graph $G$ can be represented as a string $w \in \{0,1\}^*$ where $w$ is the flattened adjacency matrix representation of $G$. In theoretical computer science we use the word ``computational problem''(decision problem) and ``language'' synonymously. So the problem of deciding if a given graph is connected or not can naturally be expressed as a language as follows:
    \textsc{connected} $ = \{w \in \{0,1\}^*| w$ is a adjacency matrix representation of a graph $G = (V,E)$ and $G$ is connected$\}$. Show that \textsc{connected} is not regular. Do you think there can be a more powerful computational model that can recognize \textsc{connected}? 
    % Enter your solution below
    \begin{solution}

    \end{solution}
  
\end{questions}
\end{document}

%%% Local Variables:
%%% mode: latex
%%% TeX-master: t
%%% End:
